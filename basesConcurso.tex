\documentclass{article}

% Language setting
% Replace `english' with e.g. `spanish' to change the document language
\usepackage[spanish]{babel}
% This package works to set lists
\usepackage{enumitem}
\date{}
% Set page size and margins
% Replace `letterpaper' with `a4paper' for UK/EU standard size
\usepackage[a4paper,top=2cm,bottom=2cm,left=2cm,right=2cm,marginparwidth=1.75cm]{geometry}

% Useful packages
\usepackage{amsmath}
\usepackage{graphicx}
\usepackage[colorlinks=true, allcolors=blue]{hyperref}
\graphicspath{{images/}}
\title{Bases del Concurso de Programación}
\author{\textbf{LlamaCode 2024-I}}

\begin{document}
\maketitle
\begin{center}
    \includegraphics[width=3cm, height=3cm]{logo-club.png}
    \includegraphics[width=2cm, height=3cm]{llamafood.png}
\end{center}
\section{Introducción}

En este concurso, tendrán la oportunidad de poner a prueba sus habilidades técnicas, resolver desafíos y demostrar su destreza en el campo de la codificación.A lo largo de esta competencia, se enfrentarán a problemas interesantes que pondrán a prueba su capacidad para pensar de manera lógica y encontrar soluciones efectivas. Este es un espacio donde podrán expandir sus conocimientos, mejorar sus habilidades y colaborar con otros entusiastas de la programación.


\section{Objetivos}
\begin{enumerate}
    \item Impulsar la capacidad analítica y comprensiva en la solución de problemas mediante algoritmos eficientes.
    \item Generar experiencias divertidas y de competitividad entre los estudiantes, a su vez haciendo networking.
    \item Promover el desarrollo de las habilidades de programación.
    \item Incentivar la participación de la comunidad estudiantil en concursos de programación \textbf{nacional e internacional}.
\end{enumerate}

\section{Reglas}
\begin{itemize}
    \item Lenguajes Permitidos:
          \begin{enumerate}
              \item C++
              \item Java
              \item C\#
              \item Python
              \item PHP
          \end{enumerate}
    \item Reglas Del Concurso
          \begin{enumerate}
              \item Tendrá una duración de 2 horas y 30 minutos.
              \item El participante no podrá comunicarse con los demás participantes, excepto alguna duda extraordinaria a los \textbf{organizadores}.
              \item Los participantes pueden llevar su laptop, en caso de que no cuenten con una, se les asignará una computadora que estará equipada con las herramientas necesarias para la resolución de los problemas.
              \item Los participantes deberán ingresar 15 minutos antes del inicio del concurso para dar más detalles y verificar credenciales.
              \item Los participantes no pueden usar recursos en línea, o el uso de IA para resolver los problemas. Cualquier participante encontrado haciendo esto será automáticamente descalificado.
              \item Los resultados de los ganadores se harán el mismo día.
              \item \textbf{Los miembros de la organización del club y desarrolladores de problemas no podrán participar.}
              \item Al finalizar el concurso se hará una pequeña sesión en la que se explicará como se resolvían los problemas.
          \end{enumerate}
          \section{Inscripción}
          La inscripción es totalmente gratuita, teniendo como plazo desde el \textbf{16 de
              mayo hasta el 21 de mayo 11:59pm.}
          \begin{enumerate}
              \item Link de inscripción: \url{https://docs.google.com/forms/d/e/1FAIpQLSdiiJGFwDP333-ZyrkZfszwf1n_zFb7UQVNPuOcovEuCmgPnw/viewform}
          \end{enumerate}
          \section{Sobre el Proceso del Concurso}
          \begin{enumerate}
              \item Consta de una fase online, la cual se dará en la plataforma OmgegaUp \url{https://omegaup.com/} con los concursantes en el lugar establecido.
              \item Los problemas serán propuestos por los miembros del comité elaborador de problemas, el mismo que establecerá los archivos oficiales de entrada y salida esperados para cada problema.
              \item Adicionalmente es necesario subir el código fuente utilizado, para efectos de
                    validación de las soluciones y detección de plagios.
              \item Los jurados solo evalúan las salidas enviadas por los participantes a través
                    de la plataforma utilizada.
              \item El jurado proporciona uno de los siguientes veredictos:
                    \begin{itemize}
                        \item \textbf{Problema Correcto}
                        \item \textbf{Error en la Respuesta.}
                    \end{itemize}
              \item Existen dos criterios para la asignación de puntaje, en orden de prioridad:
                    \begin{itemize}
                        \item \textbf{El número de problemas resueltos.}
                        \item \textbf{El tiempo total utilizado.}
                    \end{itemize}
              \item Tiempo mínimo en minutos que debe esperar un concursante después de realizar un envío para hacer otro \textbf{mínimo 1 minuto}.
          \end{enumerate}
          \section{Sobre el Calentamiento - PreContest}
          El calentamiento antes del concurso oficial se llevará acabo el \textbf{Sabado 25 de Mayo} de forma online por la plataforma \url{https://omegaup.com/} de 8:00 pm - 10:30 pm.
          \section{Comité Elaborador de Preguntas. Está Conformado Por:}
          \begin{itemize}
              \item \textbf{Oblitas Gavidia Jhon Anthony}
              \item \textbf{Quiquia Estrada Luis Franco}
              \item \textbf{Rojas Retuerto Miguel Aimar}
          \end{itemize}
          \section{Comité Organizador:}
          \begin{itemize}
              \item \textbf{Arévalo Nazario Matias Diego}
              \item \textbf{Asencios Konno Kevin}
              \item \textbf{Luyo Villalobos Mariana Valentina}
              \item \textbf{Oblitas Gavidia Jhon Anthony}
              \item \textbf{Quiquia Estrada Luis Franco}
              \item \textbf{Rojas Retuerto Miguel Aimar}
          \end{itemize}
          \section{Sobre las Sanciones:}
          \begin{enumerate}
              \item Los cóncursantes que no cumplan con las normas del presente reglamento
                    serán descalificados sin derecho a reclamos.
              \item La descalificación de un concursante implica la descalificación asignándole
                    puntaje cero.
          \end{enumerate}
          \section{Premiación:}
          \begin{enumerate}
              \item Los premios se otorgarán a los tres primeros lugares, basándose en la cantidad total de
                    problemas resueltos y el tiempo total utilizado:
                    \begin{itemize}
                        \item El primer lugar se hará acreedor de \textbf{S/.300.00 LlamaPoints}
                        \item El segundo lugar se hará acreedor de \textbf{S/.200.00 LlamaPoints}
                        \item El tercer lugar se hará acreedor de \textbf{S/.100.00 en LlamaPoints}
                    \end{itemize}
          \end{enumerate}
          \section{Fechas Importantes}
          \begin{enumerate}
              \item Fecha Calentamiento: Sábado 25 de Mayo del 2024.
              \item Fecha del Concurso: Viernes 31 de Mayo del 2024.
          \end{enumerate}
          \section{Disposiciones complementarias}
          En caso de cualquier accidente en el concurso, este se suspenderá o anulará de
          acuerdo al criterio de la comisión organizadora.



\end{itemize}

\end{document}
